

\documentclass[a4paper]{jpconf}
\usepackage{graphicx}
\usepackage{slashed}

\newcommand{\unit}[1]{\ensuremath{\mathrm{~#1}}}
\newcommand{\wjets}[0]{\mathrm{W+jets}}
\newcommand{\particle}[1]{\ensuremath{#1}}
\newcommand{\ttbar}[0]{\ensuremath{\mathrm{t\bar{t}}}}
\newcommand{\costheta}[0]{\cos\theta_{\mathrm{l,q}}^{\mathrm{(top)}}}
\newcommand{\pT}[0]{\ensuremath{p_\mathrm{T}}}
\newcommand{\mtw}[0]{\ensuremath{M_\mathrm{T}(W)}}
\newcommand{\met}[0]{\ensuremath{\slashed E_\mathrm{T}}}
\begin{document}


\title{Measurement of Top-Quark Polarization in t-channel Single-Top Production}

\author{Matthias Komm}

\address{Centre for Cosmology, Particle Physics and Phenomenology, Universit\'e catholique de Louvain, 1348 Louvain-la-Neuve, Belgium}

\ead{Matthias.Komm@CERN.ch}

\begin{abstract}
The measurement of the top-quark polarization, senstivity to the electroweak coupling structure, in single-top production via t-channel is presented. Events are analyzed, corresponding to an integrated luminosity of approximately $20\unit{fb^{-1}}$ recorded with the CMS detector during pp-collisions at $\sqrt{s}=8\unit{TeV}$ . By requiring one isolated lepton (muon or electron), two jets, and missing transverse energy, an angular distribution, sensitive to the  polarization of the top quark, is reconstructed in the top-quark rest frame. The corresponding angular distribution at parton level is inferred by unfolding from a phase space with enhanced single-top t-channel candidates. Remaining background contributions are estimated through a ML-fit and subtracted. A polarization of $P_{t}=0.82\pm0.12\mathrm{~(stat.)}\pm0.32\mathrm{~(syst.)}$ is measured assuming a spin-analyzing power of the charged lepton stemming from the top decay to be $100\%$.
\end{abstract}

\section{Introduction}
In the theory of particle physics, the Standard Model (SM), electroweak interactions between fermions via charged currents are maximally parity violating. Only left-handed fermions (or right-handed anti-fermions) can couple to W-bosons through a V-A coupling structure.


The top quark offers an unique possibility amongst all quarks to probe this prediction because of its very short lifetime below the hadronization scale. Therefore, its spin orientation is not lost through gluon radiation but stays encoded in the angular distribution of its decay products.


A observable sensitive to the electroweak top quark coupling structure is given in t-channel single top-quark production by the forward-backward asymmetry
\begin{equation}
A=\frac{N(\costheta>0)-N(\costheta<0)}{N(\costheta>0)+N(\costheta<0)}=\frac{1}{2}P_{t}\alpha_{l}
\end{equation}
in the top-quark rest frame. $\costheta$ denotes the angle between the lepton and the so-called light ($\particle{u}$,$\particle{d}$,$\particle{s}$,$\particle{c}$) quark which may also be refered to as spectator quark. The polarization, $P_{t}$, denotes the alignment of the top-quark spin with the light-quark momentum and the spin-analyzing power, $\alpha_{l}$, quantifies the alignment of lepton with the top-quark spin. Theoretical calculations show that the particular V-A structure leads to a high polarization $P_{t}=0.98$ and spin analyzing power $\alpha_{l}=1.0$ at LO~\cite{bernreuther}.


Beyond verifying the theoretical prediction, potential new particles or interactions beyond the SM might preferably occur in the top quark sector because the top quark is the heaviest known elementary particle, its mass of $m_{\mathrm{(top)}}=???\unit{GeV}$ is close to the electroweak symmetry breaking scale, and it gives major contributions in loop corrections to the Higgs self-energy. Such new physics which modify the electroweak top quark coupling structure can be characterized in their low energy limit in terms of an effective field theory (EFT)~\cite{jaaswpol}. New dimension six operators give rise to new right-handed vector or new left- and right-handed tensor couplings at the $\particle{Wtb}$-vertex which can describe modifed a spin-analyzing power and polarization. These anomalous couplings can be present at the top quark production and decay vertex. Thus they may be probed by eg. measuring the W-boson helicity fractions which focuses on the decay vertex only. For the measurement of the polarization however it has to be pointed out that beyond this EFT scenario it is furthermore sensitive to additional anonmalous couplings which are only possible in the production such as contact-interactions~\cite{fabian}.


\section{Event selection}
This analysis is based on data recorded with the CMS detector in 2012 during pp-collisions at a center-of-mass energy of $\sqrt(s)=8\unit{TeV}$ which corresponds to an integrated luminositity of $19.7\pm0.5\unit{fb^{-1}}$. Triggers looking for an isolated muon (electron) with $\pT>24~(27)\unit{GeV}$ are used.


Single top t-channel events and background processes ($\wjets$, $\ttbar$-pair production, Drell-Yan, and Di-Boson production) have been simulated using various event generators (\textsc{Powheg},\textsc{Pythia},\textsc{Sherpa}, \textsc{CompHep}) which were interfaced with \textsc{Tauola} for $\tau$ decays and \textsc{Pythia} for hardonization and matching. Finally, the simulated events are passed through the \textsc{Geant4}-based CMS detector simulation and its reconstruction software. The shape and rate of QCD multijet processes are estimated from data.


Events for analysis are required to contain one isolated muon (or electron) with $\pT>26~(30)\unit{GeV}$ and $|\eta|<2.1~(2.5)$, and two jets with $\pT>40\unit{GeV}$ where one is b-tagged using the track-counting algorithm. To reject large contribution from QCD multijet processes an additional selection on the transverse W-boson mass of $\mtw>40\unit{GeV}$ for the muon channel and on the missing transverse energy $\met>50\unit{GeV}$ for the electron channel is applied. Furthermore, a BDT has been deployed to further select a signal-enhanced phase space. Its input variables were carefully selected to be uncorrelated to $\costheta$ in order to minimize a bias torwards the sample used for training and its coupling scenario.


Finally, the top quark is reconstructed by solving the neutrino $p_{z}$-momentum using the W-boson mass constrain.



\begin{figure}[h]
\begin{center}
\begin{minipage}{7cm}
\includegraphics[height=6.5cm]{mva_bdt_mu-crop}
\center (a)
\end{minipage}\hspace{1cm}%
\begin{minipage}{7cm}
\includegraphics[height=6.5cm]{mva_bdt_el-crop}
\center (b)
\end{minipage} 
\caption{Output of the BDT trained to separate signal from $\wjets$ and $\ttbar$ in the muon (a) and electron (b) channel.}
\end{center}
\end{figure}

\begin{figure}[h]
\begin{center}
\begin{minipage}{7cm}
\includegraphics[height=6.5cm]{2j1t_cosTheta_mu-crop}
\caption{\label{fig:cosTheta_mu}Figure caption for first of two sided figures.}
\end{minipage}\hspace{1cm}%
\begin{minipage}{7cm}
\includegraphics[height=6.5cm]{2j1t_cosTheta_el-crop}
\caption{\label{fig:cosTheta_el}}
\end{minipage} 
\end{center}
\end{figure}

\section{Unfolding}

\begin{figure}[h]
\begin{center}
\begin{minipage}{7cm}
\includegraphics[height=6.5cm]{costheta_unfolded_mu-crop}
\caption{\label{fig:unfolded_mu}The unfolded $\costheta$-distribution in the muon channel.}
\end{minipage}\hspace{1cm}%
\begin{minipage}{7cm}
\includegraphics[height=6.5cm]{costheta_unfolded_el-crop}
\caption{\label{fig:unfolded_el}The unfolded $\costheta$-distribution in the electron channel.}
\end{minipage} 
\end{center}
\end{figure}

\section{Result}
\section{Conclusion}

thanks to organizers,
thanks to FNRS for funding


\section*{References}
\begin{thebibliography}{9}
\bibitem{jaaswpol} Aguilar-Saavedra J A and Bernabeu J 2010 {\it Nucl.Phys.} {\bf B840} 349-378 
\bibitem{fabian} Bach F and Ohl T 2014 {\it Phys.Rev.} {\bf D90} 074022 
\bibitem{bernreuther} Bernreuther W 2008 {\it J.Phys.} {\bf G35} 083001 


\end{thebibliography}

\end{document}


